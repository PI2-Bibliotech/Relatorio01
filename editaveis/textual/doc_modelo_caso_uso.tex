\subsection[Modelo de Caso de Uso]{Modelo de Caso de Uso}
Este documento apresenta o modelo de caso de uso, onde é desenvolvido um esquema e modelo básico das funções pretendidas do sistema e respectivamente os atuadores em seu ambiente. 

\subsubsection{Diagrama de Caso de Uso}
\begin{figure}[!h]
\centering
\includegraphics[scale=0.50, angle = 360]{figuras/caso_uso}
\caption[]{Diagrama de Caso de Uso para o projeto de software \textit{Bibliotech} (fonte: Autor)
}
\end{figure}
\FloatBarrier

\subsubsection{Descrição de Casos de Uso}

\begin{itemize}
\item{Consultar Agendamento}
\end{itemize}

Este caso de uso permite ao USUÁRIO consultar um agendamento realizado, e está consulta deve ser criada a partir do preenchimento de um formulário.

\begin{itemize}
\item{Consultar Livro}
\end{itemize}

Este caso de uso permite ao USUÁRIO consultar a disponibilidade de um livro no sistema, e está consulta deve ser criada a partir do preenchimento de um formulário.

\begin{itemize}
\item{Criar Agendamento}
\end{itemize}

Este caso de uso permite ao USUÁRIO criar um agendamento de solicitação de livro, e está consulta deve ser criada a partir do preenchimento de um formulário.

\begin{itemize}
\item{Manter Livro}
\end{itemize}

Este caso de uso permite ao BIBLIOTECÁRIO cadastrar, alterar, remover e consultar informações de um livro no sistema.

\begin{itemize}
\item{Realizar Login}
\end{itemize}

Este caso de uso permite ao BIBLIOTECÁRIO identificar-se para que eventuais permissões lhe sejam concedidas.

\begin{itemize}
\item{Manter Usuário}
\end{itemize}

Este caso de uso permite ao BIBLIOTECÁRIO cadastrar, alterar, remover e consultar informações de um USUÁRIO.

\begin{itemize}
\item{Emprestar Livro}
\end{itemize}

Este caso de uso permite ao BIBLIOTECÁRIO emprestar um livro a um USUÁRIO, e associar o empréstimo. Este empréstimo deve ser criado a partir do preenchimento de um formulário. 

\begin{itemize}
\item{Devolver Livro}
\end{itemize}

Este caso de uso permite ao BIBLIOTECÁRIO devolver um livro e finalizar o empréstimo realizado a um USUÁRIO. 

\begin{itemize}
\item{Remover Agendamento}
\end{itemize}

Este caso de uso permite ao BIBLIOTECÁRIO remover um agendamento de solicitação de livro,e está consulta deve ser criada a partir do preenchimento de um formulário.

\begin{itemize}
\item{Buscar Livro}
\end{itemize}

Este caso de uso permite ao SISTEMA \textit{WEB} solicitar ao robô a busca física de um livro em seu respectivo endereço físico.

\begin{itemize}
\item{Guardar Livro}
\end{itemize}

Este caso de uso permite ao SISTEMA \textit{WEB} solicitar ao robô devolução física de um livro em seu respectivo endereço físico.

