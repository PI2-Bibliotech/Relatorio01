\chapter[Resultados Obtidos]{Resultados Obtidos}

\section{Contagem dos Casos de Testes Implementados}
\label{sec:contagem-casos}
Pela análise do documento apresentando no anexo \ref{sec:anexo1}, observou-se
que  foram especificados 14 casos de testes. Desses 14 casos de teste, 9 casos
foram implementados. Diante disso, conclui-se que 64,3\% dos casos de testes
foram contemplados com a implementação. As listas a seguir apresentam quais casos
de teste foram implementados e quais não foram implementados.

\textbf{Casos de testes implementados}
\begin{itemize}
	\item Verificar se o sistema realiza a pesquisa por unidade de atendimento
	\item Verificar se o sistema realiza a pesquisa por especialidade do médico.
	\item Verificar o comportamento do sistema caso sejam passados dados inválidos.
	\item Verificar o comportamento do sistema caso não seja passado nenhum parâmetro.
	\item Verificar se o sistema cadastra um usuário.
	\item Verificar se o sistema permite ao usuário modificar a senha.
	\item Verificar se o usuário consegue fazer a avaliação do atendimento de um médico.
	\item Verificar se é possível remover( administrador ) comentários feitos pelos usuários.
	\item Verificar se o sistema gera relatórios sobre a quantidade de médicos por região.
\end{itemize}

\textbf{Casos de testes não implementados}
\begin{itemize}
	\item Verificar se o sistema realiza a pesquisa por nome do médico.
	\item Verificar se o sistema realiza a pesquisa por data de trabalho do médico
	\item Verificar se o sistema permite ao usuário recuperar a senha, recebendo
        um e-mail para a modificação da mesma
	\item Verificar a impossibilidade do usuário criar várias contas com o mesmo
        e-mail.
	\item Verificar se os dados disponíveis no portal da transparência do DF
        estão sendo importados corretamente para o banco de dados do TEM-DF
\end{itemize}

\section{Melhorias na qualidade dos testes}
Com os resultados obtidos a partir da contagem dos casos de testes implementados,
foi possível iniciar as melhorias na qualidade dos testes. Assim, segundo o
definido na seção~\ref{sec:qualidade-testes}, para aumentar a qualidade dos
testes de software, aumentamos a quantidade de casos de testes implementados.

Como descrito na seção~\ref{sec:contagem-casos}, 5 casos de testes não haviam
sido implementados. Assim, a equipe despendeu esforço para implementar todos
esses casos de testes. Porém, devido à complexidade de alguns casos de teste,
e ao tempo disponível para a entrega do trabalho, não foi possível implementar
todos os casos de teste. A lista a seguir, apresenta os casos de testes que
foram implementados:

\textbf{Casos de testes implementados para melhoria}
\begin{itemize}
	\item Verificar se o sistema realiza a pesquisa por nome do médico.
	\item Verificar se o sistema realiza a pesquisa por data de trabalho do
        médico
	\item Verificar a impossibilidade do usuário criar várias contas com o
        mesmo e-mail.
\end{itemize}

Diante disso, ao final do desenvolvimento do trabalho, 12 casos de teste foram
implementados. Assim a porcentagem de casos de testes implementados passou de
64,3\% para 85,7\%

\section{Integração Contínua}
\label{sec:Integração Contínua}

Como resultado da integração contínua, tivemos todas as builds de software sendo
monitoradas de acordo com os parâmetros estipulados. Assim, obtivemos maior
eficiência no momento da integração de novas builds de software para o desenvolvimento.

Fomos cobaias dessa integração, pois, a utilizamos exatamente no momento da
criação dos testes. O software tinha novas builds acopladas a cada teste implementado
de uma maneira muito mais fácil de manter pelo time responsável.

