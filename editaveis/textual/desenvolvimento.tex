\chapter[Desenvolvimento ]{Desenvolvimento}

Nesta sessão iremos responder todas as questões propostas anteriormente, sendo
que cada subsessão será responsável pela sua respectiva questão.

\section{Primeira Questão}
\label{sec:Primeira Questão}

Certamente, uma vez que ao se melhorar um processo defeituoso, ou em alguns casos estabelecer um processo formal que até então inexistente, vários problemas
de execução e gargalos no fluxo do processo são resolvidos. No caso das dificuldades essenciais, que são aquelas pertinentes à natureza do software, uma
melhoria no processo na fase de elicitação de requisitos pode prevenir que ocorram lacunas nessa tarefa que possam comprometer o produto final de software,
fazendo com que o mesmo não esteja alinhado com objetivo de negócio do cliente. Já com relação às dificuldades, uma vez que o processo esteja bem amadurecido
o impacto nesse tipo de dificuldade é bastante notável, pois com papéis e tarefas bem definidas os riscos (dificuldades) como problemas com tecnologias, ferramentas,
linguagens e assim por diante, são facilmente percebidos e consequentemente mitigados.

\section{Segunda Questão}
\label{sec:Segunda Questão}

Um exemplo de processo muito comum para o ambiente de desenvolvimento de software
é a implementação de uma história de usuário. Esta, como entrada, contém o
requisito solicitado pelo usuário e como saída, o incremento de software produzido.
A atividade é desempenhada por aquele que tem o papel de desenvolvedor, executando
assim, a implementação do requisito.

A política utilizada é que o desenvolvedor só poderá declarar a história como
concluída quando todos os critérios de aceitação estiverem cumpridos, como:
cobertura de testes, enquadramento em folha de estilo, entre outros pontos. Como
método, o desenvolvedor utilizou metodologia ágil.

\section{Terceira Questão}
\label{sec:Terceira Questão}

Um exemplo de processo imaturo é o cenário onde um desenvolvedor recebe o requisito
do cliente e o implementa, fazendo-o de maneira cíclica e sem priorização.
Neste caso, não existe nenhum controle sobre as saídas ou entradas do processo,
apenas é feito sem organização.

Um processo maduro, como exemplo, é o produto que as disciplinas de MDS/GPP
desenvolvem durante o semestre, onde é feito o planejamento das entregas, medições
dos resultados, dos incrementos de software.

\section{Quarta Questão}
\label{sec:Quarta Questão}

Nesta sessão, serão abordadas as fases para que seja possível fazer a implantação
de um projeto de Melhoria de Projeto de Software. Temos as seguintes fases a
serem pontuadas:

\begin{enumerate}
    \item Iniciação
    \item Diagnóstico
    \item Planejamento
    \item Ação
    \item Aprendizado
\end{enumerate}

Todos esses pontos serão discutidos nas subsessões a seguir:

\subsection{Iniciação}
\label{sub:Iniciação}

Nessa fase, é necessário que haja a escolha de qual contexto será aplicada a melhoria.
Bem como quais são os interessados nela. Pois se não houver nenhum patrocinador,
não é possível executar o processo a fim de melhorá-lho. Dessa maneira, há a
necessidade de uma pessoa responsável por bancar a produção

\subsection{Diagnóstico}
\label{sub:Diagnóstico}

Assim que haja fundos e recursos para investir na melhoria do projeto, a melhori
tem que ser implementada. E para isso, alguns passos são necessários, dentre eles,
a avaliação do produto do projeto por meio de um diagnóstico. E com essa fase de
análise, já podem ser levantados pontos para executar a sugestão de melhoria, ou
recomendações para tal.

\subsection{Planejamento}
\label{sub:Planejamento}

Intuitivamente, após todos os pontos serem levantados, há a necessidade de se
planejar sobre como será implementada as mudanças no projeto, bem como
definir quais são os principais pontos da mudança e desenvolver um plano de
implantação.

\subsection{Ação}
\label{sub:Ação}

Depois de analisado, sugerido e planejado, chega a desejada hora de aplicar as
mudanças no projeto, fazendo um teste piloto, após, refinando a solução e, por
fim, implementando-a de fato.

\subsection{Aprendizado}
\label{sub:Aprendizado}

Após todo processo de implantação, é feita a colheita de resultados, analisando
e avaliando o quão eficaz aquelas sujestões de melhoria foram para o projeto. E
se obtiveram êxito ou não.

Analisando os resultados finais, a equipe consegue então verificar quais melhorias
podem ser propostas para o projeto. Dessa maneira, faz-se um ciclo de melhoria
contínua onde os resultados do aprendizado servem de entrada para o diagnóstico
do processo, possibilitando assim que o desenvolvimento do software esteja em
constante melhoria.

\section{Quinta Questão}
\label{sec:Quinta Questão}

Uma avaliação de processo de software pode ser dividida em 3 grandes fases, compostas por tarefas menores.
Na primeira fase, a fase de Planejar e preparar a avaliação, uma das tarefas mais importantes é a de analisar requisitos,
pois a partir dela é possível obter um entendimento inicial do projeto para ser possível identificar os pontos aos quais
o projeto se propões a resolver.
A segunda fase, Condução da avaliação, mais expecificamente ao fim desta fase, na tarefa de gerar resultados da avaliação é
onde após todos os levantamentos terem sido feitos que obteremos a avaliação propriamente dita.
E por fim, a fase de Relatar os resultados é onde a avaliação obtida a partir da fase anterior será entregue e posteriormente arquivada.
