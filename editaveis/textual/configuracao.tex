% Definindo novas cores
\definecolor{verde}{rgb}{0,0.5,0}
% Configurando layout para mostrar codigos C++
\lstset{
  language=C++,
  basicstyle=\ttfamily\small,
  keywordstyle=\color{blue},
  stringstyle=\color{verde},
  commentstyle=\color{red},
  extendedchars=true,
  showspaces=false,
  showstringspaces=false,
  numbers=left,
  numberstyle=\tiny,
  breaklines=true,
  backgroundcolor=\color{green!10},
  breakautoindent=true,
  captionpos=b,
  xleftmargin=0pt,
}
\pagestyle{empty}

\section{Processo de configuração da Ferramenta de IC}

Como proposto nos objetivos específicos do trabalho, foi aplicada uma ferramenta de integração contínua ao projeto TEM-DF. Essa seção tem como objetivo relatar os passos, bem como
explicar cada um deles para o devido funcionamento da ferramenta em qualquer projeto.

\subsection{Configuração do Lado Github}

Primeir acesse a página de acesso aos tokens do \href{https://github.com/settings/tokens}{Github.}%
\footnote{https://github.com/settings/tokens}
Nela é possível gerar um novo token de acesso para permitir o Travis IC ter acesso ao repositório. Nas opções de permissão é necessário apenas dar permissão de acesso público ao Travis. Para isso selecione a opção (public\_repo) e gere o token.

\subsection{Configuração do Lado Travis}

Como o TEM-DF é um projeto em Ruby on Rails, utilizou-se uma gem para facilitar na configuração. Para isso utilizou-se o comando a seguir:

\begin{lstlisting}
$ gem install travis
\end{lstlisting}

Após instalar a gem é necessário criptografar o token gerado pelo Github para poder adicioná-lo no arquivo de configurações do Travis, o arquivo .travis.yml. Para isso foi executado o seguinte comando:

\begin{lstlisting}
$ travis encrypt -r user/repo 'GITHUB_SECRET_TOKEN='
\end{lstlisting}

Com o resultado deste comando, o token criptografado foi adicionado ao arquivo .travis.yml.

Também foi adicionado o caminho do script, que pode ser consultado na figura \ref{fig:sc1}, utilizado para realizar a integração automática do código. Como pode ser visto na linha 10 da figura \ref{fig:yml}. Esse script então é baixado pelo Travis e na linha 11 ele é transformado em um arquivo executável dentro do diretório /tmp.

A última configuração necessária foi informar ao travis as condições nas quais a integração contínua deveria agir.

Então configurou se uma expressão regular informando quais os casos em que o CI irá rodar, na variável BRANCHES\_TO\_MERGE\_REGEX, com o valor vazio, ou seja, rodar em todas as branchs.

Após isso, informou se em qual branch as mudanças, após atender aos critérios, irão se integradas. No caso do projeto TEM-DF a branch master.

Em seguida define se para qual repositório as integrações serão enviadas, na variável GITHUB\_REPO.

E por fim, onde devem ficar os arquivos temporários enquanto a IC estiver rodando. Sendo assim o arquivo .travis.yml ficou da seguinte forma:

Após ativar a permissão do Travis no \href{https://travis-ci.org}{endereço.}%
\footnote{https://travis-ci.org}
A integração contínua está pronta e devidamente configurada.

\begin{figure}[!ht]
  \caption{Arquivo .travis.yml}
  \label{fig:yml}

\begin{lstlisting}
language: ruby
rvm:
- 2.1.9
script:
- bundle exec rake test
env:
  global:
    secure: "Token criptografado gerado no comando anterior"
after_success:
  - "curl -o /tmp/travis-automerge https://raw.githubusercontent.com/cdown/travis-automerge/master/travis-automerge"
  - "chmod a+x /tmp/travis-automerge"
  - "BRANCHES_TO_MERGE_REGEX='' BRANCH_TO_MERGE_INTO=master GITHUB_REPO=VeV-2016-1/TEM-DF /tmp/travis-automerge"
\end{lstlisting}
\end{figure}


\begin{figure}[!ht]
  \caption{Script bash linux para auto merge.}
  \label{fig:sc1}

  \begin{lstlisting}
#!/bin/bash -e

: "${BRANCHES_TO_MERGE_REGEX?}" "${BRANCH_TO_MERGE_INTO?}"
: "${GITHUB_SECRET_TOKEN?}" "${GITHUB_REPO?}"

export GIT_COMMITTER_EMAIL='travis@travis'
export GIT_COMMITTER_NAME='Travis CI'

if ! grep -q "$BRANCHES_TO_MERGE_REGEX" <<< "$TRAVIS_BRANCH"; then
    printf "Current branch %s doesn't match regex %s, exiting\\n" \
        "$TRAVIS_BRANCH" "$BRANCHES_TO_MERGE_REGEX" >&2
    exit 0
fi

# Since Travis does a partial checkout, we need to get the whole thing
repo_temp=$(mktemp -d)
git clone "https://github.com/$GITHUB_REPO" "$repo_temp"

# shellcheck disable=SC2164
cd "$repo_temp"

printf 'Checking out %s\n' "$BRANCH_TO_MERGE_INTO" >&2
git checkout "$BRANCH_TO_MERGE_INTO"

printf 'Merging %s\n' "$TRAVIS_COMMIT" >&2
git merge --ff-only "$TRAVIS_COMMIT"

printf 'Pushing to %s\n' "$GITHUB_REPO" >&2

push_uri="https://$GITHUB_SECRET_TOKEN@github.com/$GITHUB_REPO"

# Redirect to /dev/null to avoid secret leakage
git push "$push_uri" "$BRANCH_TO_MERGE_INTO" >/dev/null 2>&1
git push "$push_uri" :"$TRAVIS_BRANCH" >/dev/null 2>&1
  \end{lstlisting}

\end{figure}
