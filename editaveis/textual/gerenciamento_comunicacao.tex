\chapter[Gerenciamento da Comunicação e Versões de Artefatos]{Gerenciamento da Comunicação e Versões de Artefatos}
\section{Ferramentas de Comunicação}
Objetivando facilitar a troca de informações entre os membros da equipe do projeto, foram escolhidas algumas ferramentas para auxílio de comunicação e controle de artefatos e anexos do projeto. 

Tendo em vista a importância e relevância de uma boa comunicação e gestão de artefatos em projetos interdisciplinares, foram selecionados mecanismos gratuitos para garantir que todos os membros do projeto tivessem acesso livre de forma espontânea.

A seguir estão apresentadas as ferramentas de comunicação e gestão adotadas e seus propósitos: 

\begin{table}[h]
\centering
\begin{tabular}{|c|p{7cm}|} \hline

Ferramenta & Propósito \\ \hline
Facebook & Espaço escolhido para se disponibilizar  links,  arquivos, discussões sobre a organização de reuniões e pequenas validações \\ \hline
WhatsApp & Comunicação básica e planejamento de reuniões \\\ hline
Hangouts & Reuniões não presenciais \\hline
Google Drive & Compartilhamento e Edição de Arquivos \\ \hline
\end{tabular}
\end{table}
\section{Reuniões}
Além das ferramentas apresentadas, foram definidas reuniões fixas e presenciais para facilitar a comunicação e execução de atividades do projeto, nos seguintes dias e horários:

Quarta-Feira: 16h às 18h 
Sexta-Feira: 14h Às 18h 

Além disso, algumas reuniões extras poderão ser realizadas caso a equipe considere necessário. Contudo, essas reuniões devem ser agendadas de forma democrática com, no mínimo, 2 dias de antecedência, pretendendo a maior presença dos membros do equipe e/ou membros que se façam necessários.
